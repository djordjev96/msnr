% !TEX encoding = UTF-8 Unicode
\documentclass[a4paper]{article}

\usepackage{color}
\usepackage{url}
\usepackage[T2A]{fontenc} % enable Cyrillic fonts
\usepackage[utf8]{inputenc} % make weird characters work
\usepackage{graphicx}
\usepackage{makecell}
\usepackage{pgfplots}

\usepackage[english,serbian]{babel}
%\usepackage[english,serbianc]{babel} %ukljuciti babel sa ovim opcijama, umesto gornjim, ukoliko se koristi cirilica

\usepackage[unicode]{hyperref}
\hypersetup{colorlinks,citecolor=green,filecolor=green,linkcolor=blue,urlcolor=blue}

\usepackage{listings}

%\newtheorem{primer}{Пример}[section] %ćirilični primer
\newtheorem{primer}{Primer}[section]

\definecolor{mygreen}{rgb}{0,0.6,0}
\definecolor{mygray}{rgb}{0.5,0.5,0.5}
\definecolor{mymauve}{rgb}{0.58,0,0.82}

\lstset{ 
  backgroundcolor=\color{white},   % choose the background color; you must add \usepackage{color} or \usepackage{xcolor}; should come as last argument
  basicstyle=\scriptsize\ttfamily,        % the size of the fonts that are used for the code
  breakatwhitespace=false,         % sets if automatic breaks should only happen at whitespace
  breaklines=true,                 % sets automatic line breaking
  captionpos=b,                    % sets the caption-position to bottom
  commentstyle=\color{mygreen},    % comment style
  deletekeywords={...},            % if you want to delete keywords from the given language
  escapeinside={\%*}{*)},          % if you want to add LaTeX within your code
  extendedchars=true,              % lets you use non-ASCII characters; for 8-bits encodings only, does not work with UTF-8
  firstnumber=1000,                % start line enumeration with line 1000
  frame=single,	                   % adds a frame around the code
  keepspaces=true,                 % keeps spaces in text, useful for keeping indentation of code (possibly needs columns=flexible)
  keywordstyle=\color{blue},       % keyword style
  language=Python,                 % the language of the code
  morekeywords={*,...},            % if you want to add more keywords to the set
  numbers=left,                    % where to put the line-numbers; possible values are (none, left, right)
  numbersep=5pt,                   % how far the line-numbers are from the code
  numberstyle=\tiny\color{mygray}, % the style that is used for the line-numbers
  rulecolor=\color{black},         % if not set, the frame-color may be changed on line-breaks within not-black text (e.g. comments (green here))
  showspaces=false,                % show spaces everywhere adding particular underscores; it overrides 'showstringspaces'
  showstringspaces=false,          % underline spaces within strings only
  showtabs=false,                  % show tabs within strings adding particular underscores
  stepnumber=2,                    % the step between two line-numbers. If it's 1, each line will be numbered
  stringstyle=\color{mymauve},     % string literal style
  tabsize=2,	                   % sets default tabsize to 2 spaces
  title=\lstname                   % show the filename of files included with \lstinputlisting; also try caption instead of title
}

\begin{document}

\title{Desavanja u Beogradu: 8 udruzenja koja organizuju redovne IT strucne sastanke (meet up) \\ \small{Seminarski rad u okviru kursa\\Metodologija stručnog i naučnog rada\\ Matematički fakultet}}

\author{Djordje Vujinovic, Nebojsa Koturovic, Igor Radojevic, Bojan Stefanovic\\ vujinovic.djordje@gmail.com, email prvog, drugog, trećeg, četvrtog autora}

%\date{1.~april 2020.}

\maketitle

\abstract{
U ovom tekstu je ukratko prikazana osnovna forma seminarskog rada. Obratite pažnju da je pored ove .pdf datoteke, u prilogu i odgovarajuća .tex datoteka, kao i .bib datoteka korišćena za generisanje literature. Na prvoj strani seminarskog rada su naslov, apstrakt i sadržaj, i to sve mora da stane na prvu stranu! Kako bi Vaš seminarski zadovoljio standarde i očekivanja, koristite uputstva i materijale sa predavanja na temu pisanja seminarskih radova. Ovo je samo šablon koji se odnosi na fizički izgled seminarskog rada (šablon koji \emph{morate} da koristite!) kao i par tehničkih pomoćnih uputstava. Pročitajte tekst pažljivo jer on sadrži i važne informacije vezane za zahteve obima i karakteristika seminarskog rada.}

\tableofcontents

\newpage

\section{Uvod}
\label{sec:uvod}

\section{PHP Serbia Meetup}
\begin{figure}[h!]
\begin{center}
\includegraphics[scale=0.25]{php_gray.jpg}
\end{center}
\caption{PHP Srbija logo}
\label{fig:pande}
\end{figure}
PHP je popularan skriptni programski jezik za izradu dinamičkih web stranica. Prema mnogim listama, PHP spada u top 10 najpopularnijih programskih jezika \cite{phpMostPopular}. U Srbiji takođe spada među najpopularnije jezike, što opravdava veliku zajednicu i višegodišnje okupljanje \cite{phpSerbiaPopularity}.

Prvi PHP Serbia Meetup organizovan je u julu 2015. godine u KC Gradu, i tadašnji plan je bio da se okupljanja održavaju svakog meseca. Do sada je organizovano 38 okupljanja u Beogradu, ne računajući radionice i panele. Predavači su uglavnom PHP programeri, a teme koje se obrađuju su vezane za sam jezik, alate, framework-e, kao i iskustva i probleme sa kojima programeri mogu da se susretnu.

Forma za prijavljivanje u svojstvu predavača se može naći na zvaničnom sajtu PHP Srbije www.phpsrbija.rs. Prijavljivanje u svojstvu slušaoca je besplatno, samo je potrebno prijaviti se na meetup.com/PHPSrbija.

Iza samog okupljanja stoji organizacija PHP Srbija koja je zaslužna i za PHP Srbija konferenciju koja je dovele neke od najboljih svetskih predavača, kao i kreatora programskog jezika PHP Rasmusa Lerdorfa \cite{phpRasmusLerdorf}. \\

\begin{tikzpicture}
\begin{axis}[
    title={Posećenost PHP Serbia Meetup-a},
    xlabel={Redni broj okupljanja},
    ylabel={Broj posetilaca},
    xmin=0, xmax=40,
    ymin=0, ymax=180,
    xtick={0,5,10,15,20,25,30,35,40},
    ytick={0,20,40,60,80,100,120,140,160},
    legend pos=north west,
    ymajorgrids=true
]

\addplot[
    color=blue,
    mark=square,
    ]
    coordinates {
    (1,55)(2,83)(3,86)(4,113)(5,110)(6,89)(7,45)(8,72)(9,66)(10,82)(11,64)(12,94)(13,82)(14,35)(15,73)(16,137)(17,165)(18,137)(19,133)(20,159)(21,116)(22,96)(23,137)(24,56)(25,139)(26,49)(27,91)(28,63)(29,95)(30,80)(31,62)(32,35)(33,88)(34,59)(35,41)(36,50)(37,88)(38,42)
    };
    
\end{axis}
\end{tikzpicture}


\section{1 meet-up}
\begin{verbatim}
š i č i ć ... 
\end{verbatim}


\section{Zaključak}
\label{sec:zakljucak}

\addcontentsline{toc}{section}{Literatura}
\appendix

\appendix
\section{Dodatak}

\end{document}
