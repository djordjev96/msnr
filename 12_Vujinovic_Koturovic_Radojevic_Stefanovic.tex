% !TEX encoding = UTF-8 Unicode
\documentclass[a4paper]{article}

\usepackage{color}
\usepackage{url}
\def\UrlBreaks{\do\/\do-}
\usepackage[T2A]{fontenc} % enable Cyrillic fonts
\usepackage[utf8]{inputenc} % make weird characters work
\usepackage{graphicx}
\usepackage{makecell}
\usepackage{pgfplots}

\usepackage[english,serbian]{babel}
%\usepackage[english,serbianc]{babel} %ukljuciti babel sa ovim opcijama, umesto gornjim, ukoliko se koristi cirilica

\usepackage[unicode]{hyperref}
\hypersetup{colorlinks,citecolor=green,filecolor=green,linkcolor=blue,urlcolor=blue}

\usepackage{listings}

%\newtheorem{primer}{Пример}[section] %ćirilični primer
\newtheorem{primer}{Primer}[section]

\definecolor{mygreen}{rgb}{0,0.6,0}
\definecolor{mygray}{rgb}{0.5,0.5,0.5}
\definecolor{mymauve}{rgb}{0.58,0,0.82}

\lstset{ 
  backgroundcolor=\color{white},   % choose the background color; you must add \usepackage{color} or \usepackage{xcolor}; should come as last argument
  basicstyle=\scriptsize\ttfamily,        % the size of the fonts that are used for the code
  breakatwhitespace=false,         % sets if automatic breaks should only happen at whitespace
  breaklines=true,                 % sets automatic line breaking
  captionpos=b,                    % sets the caption-position to bottom
  commentstyle=\color{mygreen},    % comment style
  deletekeywords={...},            % if you want to delete keywords from the given language
  escapeinside={\%*}{*)},          % if you want to add LaTeX within your code
  extendedchars=true,              % lets you use non-ASCII characters; for 8-bits encodings only, does not work with UTF-8
  firstnumber=1000,                % start line enumeration with line 1000
  frame=single,	                   % adds a frame around the code
  keepspaces=true,                 % keeps spaces in text, useful for keeping indentation of code (possibly needs columns=flexible)
  keywordstyle=\color{blue},       % keyword style
  language=Python,                 % the language of the code
  morekeywords={*,...},            % if you want to add more keywords to the set
  numbers=left,                    % where to put the line-numbers; possible values are (none, left, right)
  numbersep=5pt,                   % how far the line-numbers are from the code
  numberstyle=\tiny\color{mygray}, % the style that is used for the line-numbers
  rulecolor=\color{black},         % if not set, the frame-color may be changed on line-breaks within not-black text (e.g. comments (green here))
  showspaces=false,                % show spaces everywhere adding particular underscores; it overrides 'showstringspaces'
  showstringspaces=false,          % underline spaces within strings only
  showtabs=false,                  % show tabs within strings adding particular underscores
  stepnumber=2,                    % the step between two line-numbers. If it's 1, each line will be numbered
  stringstyle=\color{mymauve},     % string literal style
  tabsize=2,	                   % sets default tabsize to 2 spaces
  title=\lstname                   % show the filename of files included with \lstinputlisting; also try caption instead of title
}

\begin{document}

\title{Desavanja u Beogradu: 8 udruzenja koja organizuju redovne IT strucne sastanke (meet up) \\ \small{Seminarski rad u okviru kursa\\Metodologija stručnog i naučnog rada\\ Matematički fakultet}}

\author{Djordje Vujinović, Nebojša Koturović, Igor Radojević, Bojan Stefanović\\ vujinovic.djordje@gmail.com, mi15139@alas.matf.bg.ac.rs, \\ mi18493@alas.matf.bg.ac.rs, mi15022@alas.matf.bg.ac.rs}

%\date{1.~april 2020.}

\maketitle

\abstract{
%TODO(sažetak....)

\tableofcontents

\newpage
\section{Uvod}
%TODO(uvod)

\section{PHP Serbia Meetup}
\begin{figure}[h!]
\begin{center}
\includegraphics[scale=0.25]{php_gray.jpg}
\end{center}
\caption{PHP Srbija logo}
\label{fig:pande}
\end{figure}
PHP je popularan skriptni programski jezik za izradu dinamičkih web stranica. Prema mnogim listama, PHP spada u top 10 najpopularnijih programskih jezika \cite{phpMostPopular}. U Srbiji takođe spada među najpopularnije jezike, što opravdava veliku zajednicu i višegodišnje okupljanje \cite{phpSerbiaPopularity}.

Prvi PHP Serbia Meetup organizovan je u julu 2015. godine u KC Gradu, i tadašnji plan je bio da se okupljanja održavaju svakog meseca. Do sada je organizovano 38 okupljanja u Beogradu, ne računajući radionice i panele. Predavači su uglavnom PHP programeri i sistem administratori, a teme koje se obrađuju su vezane za sam jezik, alate, framework-e, kao i iskustva i probleme sa kojima programeri mogu da se susretnu. Neke teme i predavače možete videti u Tabeli \ref{tab:tabelaPHP}.  Na kraju svakog okupljanja slede pitanja posetilaca ili panel diskusija na neku temu.\\
Broj posetilaca na okupljanjima varira, a na nekima je prešao broj od 160. Posećenost na okupljanjima možete videti na grafikonu ispod \cite{phpEvents}. 

Iza samog okupljanja stoji organizacija PHP Srbija koja je zaslužna i za PHP Srbija konferenciju koja je dovele neke od najboljih svetskih predavača, kao i kreatora programskog jezika PHP Rasmusa Lerdorfa \cite{phpRasmusLerdorf}.

Forma za prijavljivanje u svojstvu predavača se može naći na zvaničnom sajtu PHP Srbije \url{https://phpsrbija.rs/}. Prijavljivanje u svojstvu slušaoca je besplatno, samo je potrebno prijaviti se na \url{https://meetup.com/PHPSrbija}.

Iz ovog okupljanja stvorilo se još nekoliko samostalnih kao na primer Laravel Serbia Meetup. Na PHP Srbija youtube kanalu se mogu naći predavanja sa okupljanja i konferencija, tako da svako ko nije bio u mogućnosti da dođe može da odgleda. Adresa youtube kanala je: \url{https://www.youtube.com/user/PHPSrbijaVideo/}.


\begin{table}[h!]
\begin{center}
\caption{Neke od tema sa okupljanja.}
\begin{tabular}{|l|l|l|} \hline
\thead{Tema} & Predavač&\#. okupljanje\\ \hline
FigDice template engine&NIkola Posa&1\\ \hline
JMeter i Performance Testing&Nebojša Videnov&6\\ \hline
\makecell[l]{Beginner talk: The road to\\become a junior developer}&\makecell[l]{Vladimir Živadinović, \\ Operations manager}&17\\ \hline
\makecell[l]{PHP Aplikacije u \\produkcionom okruženju}&\makecell[l]{Nikola Krgović,\\Sistem administrator}&28\\ \hline
\makecell[l]{Razvoj efikasnih API servisa - \\Laravel i GraphQL}&\makecell[l]{Peđa Jevtić,\\Full stack developer}&37\\ \hline
\end{tabular}
\label{tab:tabelaPHP}
\end{center}
\end{table}

\begin{tikzpicture}
\begin{axis}[
    title={Posećenost PHP Serbia Meetup-a},
    xlabel={Redni broj okupljanja},
    ylabel={Broj posetilaca},
    xmin=0, xmax=40,
    ymin=0, ymax=180,
    xtick={0,5,10,15,20,25,30,35,40},
    ytick={0,20,40,60,80,100,120,140,160},
    legend pos=north west,
    ymajorgrids=true
]

\addplot[
    color=blue,
    mark=square,
    ]
    coordinates {
    (1,55)(2,83)(3,86)(4,113)(5,110)(6,89)(7,45)(8,72)(9,66)(10,82)(11,64)(12,94)(13,82)(14,35)(15,73)(16,137)(17,165)(18,137)(19,133)(20,159)(21,116)(22,96)(23,137)(24,56)(25,139)(26,49)(27,91)(28,63)(29,95)(30,80)(31,62)(32,35)(33,88)(34,59)(35,41)(36,50)(37,88)(38,42)
    };
    
\end{axis}
\label{fig:M1}
\end{tikzpicture}

\section{WordPress Serbia Meetup}
\begin{figure}[h!]
\begin{center}
\includegraphics[scale=0.5]{wp.jpg}
\end{center}
\caption{WordPress Serbia logo}
\label{fig:pande}
\end{figure}
WordPress predstavlja jedan od najpopularnijih sistema za upravljanje sadržajem (eng.~{\em Content Management System - CMS}). Zbog lakoće podešavanja, brzog rasta platforme, mogućnosti kreiranja dodataka i tema kao i korišćenje velikog broja besplatnih iz njihovog marketa, ubrzo je pridobio pažnju programera i entuzijasta u Srbiji. WordPress Serbia je organizacija koja za cilj ima da ih sve okupi na jednom mestu. 

Prvi WordPress Serbia Meetup održan je u aprilu 2013. godine. Sledeće godine održavaju se ukupno 4 okupljanja, od čega su prva tri održana u Mikser House-u, dok je poslednji održan u Domu omladine u tribinskoj sali. Tada počinje velika zainteresovanost, tako da se prelazi u veliku salu Doma omladine. Ubrzo nastaje potreba širenja zajednice, tako da okupljanja počinju i u Nišu, Inđiji, Šapcu, Boru, Vranju, Kruševcu, Subotici, Novom Sadu, Zrenjaninu itd. Pored okupljanja kreću i sa radionicama za početnike koje su namenjene svima koji žele da nauče više o WordPress-u.

Teme koje se obrađuju na okupljanjima su prilagođene početnicima, programerima, kao i ostatku IT zajednice. Neke od tema sa okupljanja se nalaze u Tabeli \ref{tab:tabelaWordpress}. \\ \\
\begin{table}[h!]
\begin{center}
\caption{Neke od tema sa okupljanja.}
\begin{tabular}{|l|l|l|} \hline
\thead{Tema} & Predavač& \#. okupljanje\\ \hline
WordPress security & \makecell[l]{Predrag Cujanovic - \\CEO, CyberTec Security}&2\\ \hline
\makecell[l]{Scaling WordPress \\with Amazon cloud} &\makecell[l]{Miljenko Rebernisak,\\Devana Technologies}&9\\ \hline %TODO(nkoturovic: Ovde je bio neki cudan karakter zbog kog nije htelo da se kompajlira!)
\makecell[l]{Optimizacija veb-sajta za \\pretraživače - Pravila za \\dobru optimizaciju} &Stevica Gološin&20\\ \hline
\end{tabular}
\label{tab:tabelaWordpress}
\end{center}
\end{table}


Predavači su najčešće programeri, firme koje se bave izradom tema i dodataka za WordPress, dizajneri, kao i sistem administratori. Za predavača može da se prijavi bilo ko ko želi da podeli svoje iskustvo i znanje. 

Prijavljivanje za predavača je putem google forme na adresi \url{https://docs.google.com/forms/d/1kHF5Vi-35FkmAkRzdsHiQOoKrIPGOjKgVRE8mAgjA-g}. Prijavljivanje u svojstvu slušaoca je besplatno, potrebno je samo da se na adresi \url{https://www.meetup.com/WordPress-Serbia/} prijavi na željeni događaj.

Pored Meetup-a, WordPress Serbia organizuje i WordCamp konferencije na kojima gostuju svetski poznati predavači, kao i posetioci iz čitave Evrope i sveta \cite{wpWordCamp}.


\section{Python Belgrade Meetup}
\label{sec:pybgd}

\begin{figure}[h]
  \centering
  \includegraphics[width=0.3\textwidth]{pybgd.png}
  \caption{Python Belgrade Logo}
\end{figure}

Python Belgrade je nevladina organizacija koja organizuje okupljanja za sve Python entuzijaste. Ova okupljanja namenjena su svim
zainteresovanima za jezik Python, pre svega programerima, ali i svim onima koji žele da upotpune svoje znanje ovog programskog jezika i njegovih mogućnosti.

Jezik Python je nesumnjivo jedan od najpopularnijih programskih jezika danas. Sudeći po istraživanju github kompanije TIOBE, jezik Python ima sigurno mesto u top 10 najpretraživanijih jezika na poznatim web pretraživačima. \cite{pythonPopular}.

Kako i samo ime organizacije kaže, sastanci se organizuju na teritoriji grada Beograda. Ova organizacija nema svoje prostorije, a lokacija, kao i ostale informacije
za sledeći susret dostupne su na oficijalnom vebsajtu organizacije (\url{https://pythonbelgrade.com/}). Ova oraganizacija je svakako jedna od značajnijih IT organizacija koja organizuje okupljanja na teritoriji Beograda. Osnovana je 2015 godine i od tada se predavanja održavaju svakih par meseci bez striktnog međuintervala. 

Na čelu ove organizacije, ispred grupe mladih ljudi, je njen domaćin i osnivač \textit{Bojan Jovanović}. Bojan je mladi entuzijasta koji je i sam imao dosta radnog iskustva u ovom jeziku nakon završenih studija na
Elektrotehničkom fakultetu. Pored organizovanja Python Belgrade okupljanja, on je i jedan od osnivača PyCon Balkan (\url{https://pyconbalkan.com/}). Ova konferencija važi za jednu od najpoznatinjih konferencija vezanih za programiranje na ovim prostorima.

Predavači na Python Belgrade okupljanjima su najčešće ljudi koji su zaposleni ili imaju neke veze sa IT sektorom, ali neretko i ljudi sa univerziteta. \cite{pybgdSpeakers} Iako se predavanja održavaju na srpskom jeziku, dešavalo se da zbog prisustva stranih slušalaca u duhu solidarnosti bude napravljen izuzetak i predavanje se održi na engleskom jeziku. Na okupljanjima se nesumnjivo oseća mladalači duh i želja za napredovanjem. Uvek vlada prijateljska atmosfera, nađu se tu grickalice i osveženje, pa nije ni čudo što se okupljeni između sebe zovu ``Pytosi''.

Putem vebsajta \url{https://pythonbelgrade.com/} se može prijaviti za prisustvo nekom od okupljanja, a obaveštenja se objavljuju i na fejsbuk grupi \href{https://www.facebook.com/python.belgrade/}{@python.belgrade} i \href{https://pythonbelgrade.slack.com/join/shared_invite/enQtNTM0OTI4NzY3NDI2LTk2OThkNjQ2YmM5NWNkNTBjMjU3NDY0NjcyNDFiYmZmNjQ4MDE3NjczYWM0NTk1YWVlYzlhNGIwYjU3NjA3Y2Y}{slack kanalu}. Svako ko želi da se prijavi u svojstvu predavača potrebno je da popuni odgovarajuću web formu za prijavu na \ref{https://pythonbelgrade.com/}{vebsajtu} organizacije (\url{https://pythonbelgrade.com/#contact}). Predavanja se mogu pratiti u putem video linka uživo, gde se takođe mogu postavljati pitanja. Neka od već održanih predavanja dostupna su na YouTube kanalu ove organizacije (\url{https://www.youtube.com/channel/UCH79IYl8rv2f2BrYs5owB2A}).

\section{Belgrade Crypto Community}
\label{sec:bgdcs}

\begin{figure}[h]
  \centering
  \includegraphics[width=0.3\textwidth]{bcc_logo.png}
  \caption{Belgrade Crypto Community logo}
\end{figure}

Termini  kriptovaluta, blokčejn, majner, digitalni novčanik i bitkoin su od svog nastanka zaintrigirali i dobili pažnju ne samo IT scene, već i javnosti. To je bilo lakši deo, dok tačno definisanje pojmova, koncepta i relacija medju njima baš i ne. Belgrade Crypto Community je organizacija sa ciljem da ovu tematiku popularizuje i prenese svoje znanje na što veći broj ljudi.

Belgrade Crypto je domaća kripto zajednica koja povezuje više od 1200 članova, od korisnika i novajlija do iskusnih programera blokčejna. Prostorije im se nalaze u Balkanskoj 2, gde se održavaju mitapovi, dobro su opremljene i imaju kapaciteta za veći broj učesnika. Informacije o samoj zajednici kao i buduće susrete možete naći na zvaničnom veb-sajtu \url{https://belgradecrypto.com/}. Na sajtu postoji blog koji se bavi zanimljivim i aktuelnim temama.

Prvi Belgrade Crypto Meetup održan je u februaru 2017. godine. Od tad su promenili više lokacija i formata samog mitapa, trenutno se održavaju svakog drugog četvrtka u mesecu od 18 do 20 časova. U planu je organizacija i u drugim gradovima, a prvi takav je bio u Novom Sadu. Mitapovi se obično realizuju na jedan od dva načina: 1. opušteno druženje, pričanje o aktulenostima i novitetima; 2.predavanje eksperata. Bilo je i nekoliko radionica, na kojima su učesnici mogli  da se upoznaju sa lokalnom berzom Xcalibra, globalnom platformom za trgovinu Safex i automatom za kupovinom kritovaluta HashBringer. 

Forme za prijavu u svojstvo predavaca ili slusaoca je moguče naći na stranici \url{https://www.meetup.com/Belgrade-Crypto-Community/}. Belgrade Crypto Community ima stranice na društvenim mrežama \url{https://www.facebook.com/BelgradeCryptoCommunity/}, \url{https://twitter.com/Belgrade_Crypto}, \url{https://www.instagram.com/belgrade_crypto/}, \url{https://t.me/belgrade_crypto}, preko kojih informise o svojim aktivnostima. Pojedine susrete i video tutoriale možete naći na You-Tube stranici \url{https://www.youtube.com/channel/UCCv76WWtH9LppS65OCWpI4A/featured}.

\newpage
\section{Zaključak}
\label{sec:zakljucak}
%TODO()

\addcontentsline{toc}{section}{Literatura}
\appendix
\bibliography{seminarski} 
\bibliographystyle{unsrt}

\appendix
\section{Dodatak}
%TODO(izbrisati ako nam ne treba)
\end{document}
