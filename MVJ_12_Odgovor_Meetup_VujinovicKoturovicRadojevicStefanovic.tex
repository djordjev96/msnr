

 % !TEX encoding = UTF-8 Unicode

\documentclass[a4paper]{report}

\usepackage[T2A]{fontenc} % enable Cyrillic fonts
\usepackage[utf8x,utf8]{inputenc} % make weird characters work
\usepackage[serbian]{babel}
%\usepackage[english,serbianc]{babel}
\usepackage{amssymb}

\usepackage{color}
\usepackage{url}
\usepackage[unicode]{hyperref}
\hypersetup{colorlinks,citecolor=green,filecolor=green,linkcolor=blue,urlcolor=blue}

\newcommand{\odgovor}[1]{\textcolor{blue}{#1}}

\begin{document}

\title{Udruženja koja organizuju redovna IT okupljanja u Beogradu\\ \small{Đorđe Vujinović, Nebojša Koturović, Igor Radojević, Bojan Stefanović}}
\author{Đorđe Vujinović, Nebojša Koturović, Igor Radojević, Bojan Stefanović}

\maketitle

\tableofcontents
 
\chapter{Recenzent \odgovor{--- ocena: 5} }


\section{O čemu rad govori?}
% Напишете један кратак пасус у којим ћете својим речима препричати суштину рада (и тиме показати да сте рад пажљиво прочитали и разумели). Обим од 200 до 400 карактера.
Rad se bavi istraživanjem udruženja koja organizuju okupljanja u Beogradu. 
Pored kratkog istorijata i organizatora, spominju se i ciljevi organizovanja 
okupljanja svih navedenih zajednica. Za svako udruženje prikazana su i predavanja, 
njihove teme, ko su bili predavači i kojeg su datuma održana.

\section{Krupne primedbe i sugestije}
% Напишете своја запажања и конструктивне идеје шта у раду недостаје и шта би требало да се промени-измени-дода-одузме да би рад био квалитетнији.
Strukturno i estetski bi više imalo smisla da su udruženja predstavljena kao podglave, 
a ne kao zasebne glave, pošto objedinjuju jednu istu celinu.
\odgovor{Odbijamo, razmotrili smo ovu opciju i pre prve verzije rada. Ukoliko bismo sve stavili kao jednu glavu, ta glava bi nosila ime naslova rada. Jedina povezanost naših naslova je što su sve to udruženja što je i sama tema rada. Po toj logici svaki rad bi imao uvod, razradu koja bi imala ponovljen naslov rada i zaključak. A pored ovoga prebacivanje naslova u podnaslove umanjuje njihovu važnost.}
Pošto je timski rad, potrudite se da ujednačite stil pisanja rada.
\odgovor{Prihvaćeno i ispravljeno. Ubačene su tabele za svako udruženje i za razliku od prve verzije svaka je sortirana na isti način. Formatiranje datuma, url adresa je izvršeno na isti način.}
Smanjite ponavljanje istih reči u istoj rečenici.
\odgovor{Prihvaćeno i ispravljeno.}
Linkovi koji vode ka dodatnom sadržaju udruženja (kao što su \textit{Slack} kanali, \textit{YouTube} plejliste, \textit{Github} repozitorijumi, 
\textit{Meetup} događaji, \textit{Facebook} grupe) mogu da se stave fusnotu radi boljeg estetskog izgleda, dok
linkove koji vode ka zvaničnim stranicama udruženja ostavite u glavnom tekstu. 
\odgovor{Prihvaćeno i ispravljeno.}
Nemojte vezivati linkove za reči (slučaj sa \textit{Facebook} i \textit{Slack} grupom za Python Belgrade). 
\odgovor{Prihvaćeno i ispravljeno.}
Ako navodite tabele, navedite jednu zasebnu za sve, da vam ne uzima previše prostora za tekst ili
navedite manje tabele sa po tri-četiri predavanja za svako udruženje, da bi čitalac imao uvid u njih.
\odgovor{Prihvaćena druga varijanta i ispravljena.}
Ako navodite slike, više bi doprinela radu slika sa nekog okupljanja nego logo 
udruženja (čitalac svakako logo može videti kad pristupi stranicama udruženja).
\odgovor{Odbijeno, rad služi da upozna čitaoca sa udruženjima. Smatramo da logo donosi više informacija nego slika jer to udruženje posle prepoznatljivo čitaocu.}
Navesti barem jednu listu za nabrajanje.
\odgovor{Prihvaćeno i ispravljeno.}
Slike grafika koje predstavljaju odnos između broja posetilaca
i rednog broja okupljanja ne daju toliko značajnu informaciju, bolje je prikazati odnos broja posetilaca na
godišnjem nivou svih udruženja na jednoj slici, nego za svako udruženje pojedinačno.
\odgovor{Odbijeno. Grafik ne bi bio pregledan jer bi bilo previše informacija na jednoj slici.}
Podebljati reči koje smatrate bitnim. \odgovor{Odbijeno. Smatramo da zbog teme rada ovo nije potrebno.}
Produžiti uvod i zaključak koji deluju zbrzano.  \odgovor{Prihvaćeno i ispravljeno. Uvod je proširen tako sto su dodate rečenice na kraj uvoda počev od rečenice ,,Organizacije su birane tako da..''}
Literatura neka ide na zasebnoj strani, a ne zajedno sa zaključkom. \odgovor{Prihvaćeno i ispravljeno.}
Obratite pažnju na pravopis i kako navodite strane reči. \odgovor{Prihvaćeno i ispravljeno.}
Izbegavajte korišćenje engleskih reči za koje postoji adekvatan stručni prevod u srpskom jeziku. \odgovor{Prihvaćeno i ispravljeno.}

%\newpage
\section{Sitne primedbe}
% Напишете своја запажања на тему штампарских-стилских-језичких грешки
Jezičke greške: \begin{itemize}
\item Velika ekspanzija IT sektora, poslednjih godina u Beogradu, dovela je...\\
      Velika ekspanzija IT sektora poslednjih godina u Beogradu dovela je...
      \odgovor{Prihvaćeno i ispravljeno.}
\item framework -> okruženje 
\odgovor{Prihvaćeno i ispravljeno.}
\item ... "Matematičkog fakulteta" ... -> ... Matematičkog fakulteta ... 
\odgovor{Prihvaćeno i ispravljeno.}
\item Osnivači imaju bogato iskustvo... i borili se sa izazovima i manama tih metoda. \\
      Osnivači imaju bogato iskustvo... i borili su se sa izazovima i manama tih metoda.
       \odgovor{Prihvaćeno i ispravljeno.}
\end{itemize}

Stilske greške:\begin{itemize}
\item Udruženja organizujte u podglavlja, a ne zasebne glave.
\odgovor{Odgovoreno u delu za krupne primedbe.}
\item Susret (eng. meetup) ... -> \textbf{Susret} (eng. \textit{meetup}) ... .
\odgovor{Prihvaćeno sve sem boldovanja i ispravljeno.}
\item Podebljajte bitne reči u svakom podglavlju.
\odgovor{Odgovoreno u delu za krupne primedbe.}
\item Linkove koji vode ka glavnim stranama udruženja ostavite u tekstu, ostale stavite u fusnote.
\odgovor{Prihvaćeno i ispravljeno.}
\item Rad bi lepše izgledao ako bi podnaslov WordPress Serbia počeo na novoj strani.
\odgovor{Prihvaćeno i ispravljeno.}
\item Postavite logo Belgrade Crypto Community da dođe ispod podnaslova.
\odgovor{Prihvaćeno i ispravljeno.}
\item Python Belgrade sekcija ima ponavljanje istog URL linka više puta.
\odgovor{Prihvaćeno i ispravljeno.}
\item Python Belgrade sekcija ima vezane linkove za reči.
\odgovor{Prihvaćeno i ispravljeno.}
\item Skram -> Skram (eng. \textit{Scrum})
\odgovor{Prihvaćeno i ispravljeno.}
\item Slike i tabele uzimaju previše prostora koji je mogao da se iskoristi za tekst.
\odgovor{Smatramo da je tabela sa temama okupljanja jako bitna zato što čitaoca može bliže da upozna sa temama koje se obrađuju na samim okupljanjima, a slike predstavljaju grafički identitet organizacija o kojima je pisano.}

\end{itemize}

Štamparske greške:\begin{itemize}
\item \textit{Bojan Jovanović} -> Bojan Jovanović
\odgovor{Prihvaćeno i ispravljeno.}
\item Promenite vrste susreta u Belgrade Crypto Community podglavlju da budu numerisana lista.
\odgovor{Prihvaćeno i ispravljeno.}
\item U literaturi se dva puta pojavljuje isti link za JS Belgrade okupljanja, konkretno
      [\textbf{12}] i [\textbf{13}].
\odgovor{Prihvaćeno i ispravljeno. U literaturi ostavljen samo link: \url{https://www.meetup.com/JS-Belgrade-Meetup/}}
\end{itemize}



\section{Provera sadržajnosti i forme seminarskog rada}
% Oдговорите на следећа питања --- уз сваки одговор дати и образложење

\begin{enumerate}
\item Da li rad dobro odgovara na zadatu temu?\\
Rad je solidno opisao predstavnike koje je naveo, davajući kratak opis o udruženjima,
cilj njihovih okupljanja i kakve teme zainteresovani mogu očekivati. 
\item Da li je nešto važno propušteno?\\
Van pravopisnih, strukturnih i stilskih koje su navedene u primedbama, 
rad se drži teme kojom se bavi.
\item Da li ima suštinskih grešaka i propusta?\\
Potrebno je napisati objedinjujuću razradu. Malo detaljnije opisati 
svako udruženje, a ne samo navoditi linkove, slike, tabele.
\odgovor{Prihvaćeno i ispravljeno.}
\item Da li je naslov rada dobro izabran?\\
Naslov opisuje temu kojom se bavi rad, ali takođe razmatrajte kao 
opciju "7 poznatih udruženja koja organizuju redovna IT
okupljanja širom Beograda".
\odgovor{Odbijeno, smatramo da je prvobitni naslov prikladniji.}
\item Da li sažetak sadrži prave podatke o radu?\\
Da, u sažetku je u kratkim crtama predstavljena tema kojom će se
rad baviti i na koji način. Navedene ključne reči jesu ključne za rad.
\item Da li je rad lak-težak za čitanje?\\
Rad je veoma lak za čitanje. U tekstu nisu korišćene prekomplikovane reči niti rečenice.
\item Da li je za razumevanje teksta potrebno predznanje i u kolikoj meri?\\
Poželjna je osnovna upućenost u stručne IT termine, ali se i bez toga rad može razumeti.
\item Da li je u radu navedena odgovarajuća literatura?\\
Navedena literatura je odgovarajuća, pošto su uglavnom linkovi ka zvaničnim stranicama
udruženja i linkovi za određene događaje udruženja. U skladu je sa uslovom seminarskog rada. 
\item Da li su u radu reference korektno navedene?\\
Sve reference su korektno navedene i u skladu sa uslovom seminarskog rada.
Navedena je bar jedna knjiga, bar jedan naučni članak i barem jedna adekvatna veb adresa.
\item Da li je struktura rada adekvatna?\\
Rad bi strukturno izgledao bolje ako bi koristili podglavlja za opisivanje udruženja,
a da ih jedna glava sve objedinjuje. Koristiti fusnote za dodatne materijale.
Literaturu odvojiti na zasebnoj strani od zaključka.
\odgovor{Odgovoreno u delu za krupne primedbe.}
\item Da li rad sadrži sve elemente propisane uslovom seminarskog rada (slike, tabele, broj strana...)?\\
Da, rad ima od 10 do 12 strana, bar jednu sliku i bar jednu tabelu. Ima više od 7 referenci,
među kojima je jedna knjiga, bar jedna adekvatna veb adresa i bar jedan adekvatan naučni članak.
\item Da li su slike i tabele funkcionalne i adekvatne?\\
Budući da pristupom na sajt udruženja možemo videti logo, ona ne doprinosi toliko radu koliko bi recimo
slika sa nekog okupljanja. Što se grafika tiče, bilo bi bolje prikazati odnos broja posetilaca za svaku 
godinu. Adekvatnije su tabele o predavanjima nego tabele broja okupljanja kroz godine.
\odgovor{Odgovoreno u delu za krupne primedbe.}
\end{enumerate}

\section{Ocenite sebe}
% Napišite koliko ste upućeni u oblast koju recenzirate: 
% a) ekspert u datoj oblasti
% b) veoma upućeni u oblast
c) srednje upućeni\\
% d) malo upućeni 
% e) skoro neupućeni
% f) potpuno neupućeni
% Obrazložite svoju odluku
Bio sam na nekoliko okupljanja u prethodnih par godina, uglavnom
u Startit centru. Pored toga, prisustvovao sam na dva \textit{Zühlke} okupljanja.  


\chapter{Recenzent \odgovor{--- ocena: 4} }


\section{O čemu rad govori?}
% Напишете један кратак пасус у којим ћете својим речима препричати суштину рада (и тиме показати да сте рад пажљиво прочитали и разумели). Обим од 200 до 400 карактера.
Ovaj rad govori o poznatim IT udruženjima koja organizuju meetup-e u Beogradu (neka i u ostatku Srbije). Nudi osnovne informacije o 7 udruženja: osnovni cilj, kad su nastale, lokacija sastanaka kao i koliko su česti. Posećenost je prikazana graficima, a teme koje se obrađuju tabelama. Za svaku organizaciju su navedeni podaci o zvaničnoj stranici i stranicama na društvenim mrežama.

\section{Krupne primedbe i sugestije}
% Напишете своја запажања и конструктивне идеје шта у раду недостаје и шта би требало да се промени-измени-дода-одузме да би рад био квалитетнији.

Za udruženja Python Belgrade i Belgrade Crypto Community ne postoji nikakva slika ili tabela. Bilo bi dobro da je dodata tabela tema skorašnjih meetup-a ili grafik posećenosti kao što postoji za ostala udruženja.
\odgovor{Prihvaćeno. Dodata tabela u glavi Belgrade Crypto Community i Python Belgrade sa temama nekih skorašnjih susreta.}

Za neka udruženja poput PHP Serbia i WordPress Serbia nije napisan neki konkretan uvod u samo udruženje, već samo uvod u programski jezik ili slično. Savet je da se napiše nekakav uvod sličan prvom pasusu vezanom za Python Belgrade. 
\odgovor{Prihvaćeno i ispravljeno.}

Za neke česte lokacije okupljanja poput StartIt ne piše gde se tačno nalazi. Moglo bi da se objasni ili da se navede referenca na neki sajt gde te informacije mogu da se nađu. \odgovor{Prihvaćeno i ispravljeno.}

U pretposlednjem pasusu vezanom za udruženje Python Belgrade poslednja rečenica deluje malo neozbiljno za jedan seminarski rad. Sugestija je da se rečenica ili izbaci ili pokuša izmeniti tako da bude formalnija.\odgovor{Prihvaćeno i ispravljeno. Rečenica je izbačena.}

Tvrdnja da postoji ekspanzija IT sektora u Beogradu nije pokrepljena nikakvim izveštajem. Mogla bi se dodati referenca na članak koji prikazuje neku statistiku.
\odgovor{Prihvaćeno i ispravljeno. Tvrdnja je potkrepljena člankom.}

Jedan od zahteva je da se u literaturi nalazi bar jedan naučni članak iz odgovarajućeg časopisa, što ovaj rad nema. Ostali zahtevi su ispunjeni.
\odgovor{Prihvaćeno i ispravljeno. Dodat je naučni članak kod Python Belgrade (o tome da je Python jezik koji najbrže raste).}

\section{Sitne primedbe}
% Напишете своја запажања на тему штампарских-стилских-језичких грешки

Štamparske, stilske i jezičke greške sa tačnim mestom gde se nalaze u tekstu:
\begin{enumerate}
\item Uvod, prvi pasus, druga rečenica: treba zarez nakon "{seminara}"
\odgovor{Prihvaćeno i ispravljeno.}
\item Uvod, drugi pasus, peta rečenica: treba "{Dosta kompanija vidi}"
\odgovor{Prihvaćeno i ispravljeno.}
\item Python Serbia, poslednji pasus, prva rečenica: za facebook grupu i slack kanal treba staviti puno ime adrese, jer ovako neko ko čita odštampanu verziju rada ne može naći te stranice
\item Belgrade Crypto Community, prvi pasus, druga rečenica: treba "To je bio"
\odgovor{Tokom prepravke tog pasusa spomenuta rečenica je obrisana.}
\item Agile Serbia, prvi pasus, druga rečenica: treba "kad se sastalo"
\odgovor{Prihvaćeno i ispravljeno.}
\item Agile Serbia, treći pasus, prva rečenica: treba StartIt i ICT velikim slovima
\odgovor{Prihvaćeno i ispravljeno.}
\item Agile Serbia, treći pasus, prva i druga rečenica: razmak između tačke i velikog slova
\odgovor{Prihvaćeno i ispravljeno.}
\item Agile Serbia, poslednji pasus, poslednja rečenica: treba HR velikim slovima
\odgovor{Prihvaćeno i ispravljeno.}
\item Agile Serbia, poslednji pasus: ne treba da bude odvojen od poslednjeg pasusa za dva nova reda već za samo jedan
\odgovor{Prihvaćeno i ispravljeno.}
\item JS Belgrade, treći pasus, sedma rečenica: treba " slušalac "
\odgovor{Prihvaćeno i ispravljeno.}
\item JS Belgrade, poslednji pasus: nema formu rečenice
\odgovor{Prihvaćeno i ispravljeno. Pasus je dosta promenjen tako da ima formu rečenice, a linkovi u njemu su stavljeni u fusnotu}
\end{enumerate}


\section{Provera sadržajnosti i forme seminarskog rada}
% Oдговорите на следећа питања --- уз сваки одговор дати и образложење

\begin{enumerate}
\item Da li rad dobro odgovara na zadatu temu?\\
Da, za svaku organizaciju odgovorena su pitanja data u opisu teme.
\item Da li je nešto važno propušteno?\\
Nije.
\item Da li ima suštinskih grešaka i propusta?\\
Nema.
\item Da li je naslov rada dobro izabran?\\
Naslov korektno predstavlja suštinu rada.
\item Da li sažetak sadrži prave podatke o radu?\\
Sažetak navodi sve teme o kojima će se govoriti u radu za svako udruženje.
\item Da li je rad lak-težak za čitanje?\\
Rad je lak za čitanje. Svaka glava može da se čita kao celina za sebe.
\item Da li je za razumevanje teksta potrebno predznanje i u kolikoj meri?\\
Nije. Rad čitaoca uvodi u priču o udruženjima kao i o tome koja je svrha udruženja i nije potrebno biti upućen u ta udruženja pre čitanja rada.
\item Da li je u radu navedena odgovarajuća literatura?\\
Jeste, samo što su neki sajtovi navedeni u tekstu umesto da budu navedeni u literaturi.
\odgovor{Prihvaćeno i ispravljeno.}
\item Da li su u radu reference korektno navedene?\\
Jesu.
\item Da li je struktura rada adekvatna?\\
Jeste, rad je podeljen u celine po udruženjima.
\item Da li rad sadrži sve elemente propisane uslovom seminarskog rada (slike, tabele, broj strana...)?\\
Sadrži sve elemente, osim toga da jedna referenca u literaturi treba da bude naučni članak iz odgovarajućeg časopisa.
\odgovor{Prihvaćeno i ispravljeno (dodat članak u Python Belgrade sekciji).}
\item Da li su slike i tabele funkcionalne i adekvatne?\\
Da, jasno je šta predstavljaju i bez čitanja teksta.
\end{enumerate}

\section{Ocenite sebe}
% Napišite koliko ste upućeni u oblast koju recenzirate: 
% a) ekspert u datoj oblasti
% b) veoma upućeni u oblast
c) srednje upućeni\\
% d) malo upućeni 
% e) skoro neupućeni
% f) potpuno neupućeni
% Obrazložite svoju odluku
Pre recenziranja sam pročitao šta piše na sajtovima udruženja kao i na njihovim stranicama na društvenim mrežama, ali pre toga za većinu udruženja nisam čuo.


\chapter{Recenzent \odgovor{--- ocena: 4} }


\section{O čemu rad govori?}
% Напишете један кратак пасус у којим ћете својим речима препричати суштину рада (и тиме показати да сте рад пажљиво прочитали и разумели). Обим од 200 до 400 карактера.
U radu su predstaljvena udruženja PHP Srbija, WordPress Serbia, Python Belgrade, RISK, Belgrade Crypto Community, Agile Serbia i JS Belgrade, koja organizuju redovna okupljanja u Beogradu. O svakom udruženju napisane su osnovne informacije kao što su ciljevi okupljanja, neka zanimljiva predavanja koja su do sada držali, gde i kada se sastaju, koliku posećenost imaju, kao i načine za prijavljivanje kako slusanja predavanja tako i držanja. Takođe su date dalje smernice za dobijanje detaljnijih informacija.
\section{Krupne primedbe i sugestije}
% Напишете своја запажања и конструктивне идеје шта у раду недостаје и шта би требало да се промени-измени-дода-одузме да би рад био квалиjтетнији.
\begin{itemize}
    \item Potrebno je rečenice pisane u prvom licu  preformulisati u neutralan oblik. 
    Primer u sekciji Uvod: ,,U daljem tekstu predstavićemo Vam sledeće organizacije: ...'' prepraviti u ,,U daljem tekstu biće predstavljene sledeće organizacije: ...''.
    \odgovor{Prihvaćeno i ispravljeno.}
    \item Informacije u tabelama navoditi dosledno, ili rastuće po datumu održavanja ili opadajuće. Moja preporuka je opadajuće jer su od većeg značaja skorije održana predavnja. 
    \odgovor{Prihvaćeno i ispravljeno.}
    Takođe, u sekciji o WordPress-u, navesti skorija održana predavnja da se ne bi stekao utisak da organizacija nije aktivna. \odgovor{Prihvaćeno i ispravljeno.}
    \item Izbegavati iznošenje subjektivnog uverenja u radu.
    Primer u sekciji Python Belgrade, šesti pasus: ,,Na okupljanju se nesumnjivo oseća mladalački duh i želja za napredovanje. Uvek vlada prijateljska atmosfera, nađu se tu grickalice i osveženje, pa nije ni čudo što se okupljeni između sebe zovu ,,Pytosi''''.
    \odgovor{Prihvaćeno i ispravljeno.}
    \item U sekciji Belgrade Crypto Community, prvi pasus, prve dve rečenice. Ako je iznošenje ličnog stava treba ga ostaviti za zaključak, u suprotnom navesti čiji stav je u pitanju, odakle je citirano. Isto važi za rečenicu u sekciji Agile Serbia, u prvom pasusu: ,,Agilna poslovanja rade brže, bolje, sa većom vrednošću izlaznog proizvoda i sa većim procentom uspešno završenih projekaata.''. 
    \odgovor{Prihvaćeno i ispravljeno. Za Belgrade Crypto Community obrisana je druga recenica gde je bio stav autora, a za Agile Serbia ubacen je izvor.}
    \item Zaključak lepo sumira nalaze i daje uputstva za dalje istraživanje. Preporuka je samo preformulisati rečenice koje sadrže glagole ,,nadamo'', i ,potrudili'' jer njima zaključak gubi na snazi.
    \odgovor{Prihvaćeno i ispravljeno. Zaključak preformulisan u skladu sa ovim uputstvima.}
    
\end{itemize}{}
\section{Sitne primedbe}
% Напиpredшете своја запажања на тему штампарских-стилских-језичких грешки
\begin{itemize}
    \item Uvod, drugi pasus, prva rečenica - permutovana slova u reči interesovanjem.
    \odgovor{Prihvaćeno i ispravljeno.}
    \item PHP Srbija, drugi pasus - uvući rečenicu ,,Broj posetilaca na okupljanju varira, a na nekim je prešao broj od 160.''. 
    \odgovor{Prihvaćeno i ispravljeno.}
    \item WordPress:
    
    Prvi pasus - preformulisati rečenicu ,,Zbog lakoće podešavanja, brzog rasta platforme, mogućnosti kreiranja dodataka i tema kao i korišćenje velikog broja besplatnih iz njihovog marketa, ubrzo je pridobio pažnju programera i entuzijasta u Srbiji.''. Na primer: ,,Zbog svoje lakoće podešavanja, brzog rasta platforme, mogućnosti kreiranja dodataka i tema, kao i korišćenje gotovih, pridobio je pažnju programera i  entuzijasta u Srbiji.''. 
    
    Drugi pasus - u rečenici ,,Sledeće godine održavaju se ukupno 4 okupljanja, od čega su prva tri održana ...'', umesto karaktera za broj, napisati slovima četiri, da bi u rečenici dosledno bili napisani brojevi.
    
     Poslednji pasus - umesto ,,Meetup-a'' staviti ,,događaja''. \\
     \odgovor{Sve navedeno je prihvaćeno i ispravljeno.}
    \item RISK:
    
    Tabela - staviti veliko slovo u nazivu firme ,,Origintrail''.
    
    \item Belgrade Crypto Community, treći pasus:
    
    Možda bi trebalo staviti tačku umesto zareza u rečenici ,,Od tada su promenili više lokacija i formata samog okupljanja, trenutno se održavaju svakog drugog četvrtka u mesecu od 18 do 20 časova''.
    \odgovor{Prihvaćeno i ispravljeno.}
    Uskladiti vremena u rečenici ,,U planu je organizacija i u drugim gradovima, a prva takva je bio u Novom Sadu''.
    \odgovor{Prihvaćeno, izmenjena je cela rečenica.}
    Permutovana slova u reči ,,aktuelnostima''.
    \odgovor{Nije prihvaćeno, reč je lepo napisana. }
    \item Agile Serbia:
    
    Prvi pasus - Umesto ,,Agile Manifesto'' napisati ,,Agilni Manifesto''.
    \odgovor{Prihvaćeno i ispravljeno.}
    
    Drugi pasus -  preformulisati pasus, konkretno rečenice ,,Osnivači imaju bogato iskustvo sa tradicionalnim metodama razvoja softvera i borili se sa izazazovima i manama metoda. Novi prisup je bio Skram čije korišćenje je delovalo sasvim logično tako da je novi cilj pomoći drugima sa istim problemima i proširiti već postojeće znanje.'', jer je izgubljen tok misli.
    \odgovor{Delimično je prihvaćeno, jer su ispravljene samo stilske greške, a nije menjan ceo pasus.}
    
    Treći pasus - izbaciti reč ,,generalno'', ,,StartIt'' i ,,ICT centru'' napisati velikim početnim slovom, staviti tačku posle godine 2012, umesto ,,agile duhom'', napisati ,,agilnim duhom''. 
    \odgovor{Prihvaćeno i ispravljeno.}
    \item JavaScript:
    
    Prvi pasus - možda bi trebalo preformulisati drugu rečenicu tako da bude smanjen broj ponavlja glagola ,,može se''.
    \odgovor{Prihvaćeno i ispravljeno. Deo rečenice je izmenjen tako da je smanjen broj ponavljanja.}
    Drugi pasus - umesto ,,slušaoc'', ,,slušalac''
    \odgovor{Prihvaćeno i ispravljeno.}
    \item Često se u tekstu pojavljuje pridevska zamenica ,,bilo ko'',  možda bi negde trebalo da se zameni pridevskom zamenicom ,,svako''.
    \odgovor{Prihvaćena i ispravljeno.}
    \item Svuda u tekstu zameniti navodnike '' '' navodnicima ,, ''.
    \odgovor{Prihvaćeno i ispravljeno.} %TODO proveriti jos jednom
    
    %\item Kada se prvi put navodi strana reč, možda bi trebalo u zagradu staviti kako se čita, ovo se odnosi i na imena i prezimena.
   
    
\end{itemize}
\section{Provera sadržajnosti i forme seminarskog rada}
% Oдговорите на следећа питања --- уз сваки одговор дати и образложење

\begin{enumerate}
\item Da li rad dobro odgovara na zadatu temu?\\
Da, ono što stoji u radu odgovara na zadatu temu.
\item Da li je nešto važno propušteno?\\
Nije ništa važno propušteno.
\item Da li ima suštinskih grešaka i propusta?\\
Nema suštinski grešaka i propusta.
\item Da li je naslov rada dobro izabran?\\
Naslov je dobro izabran, sadrži konkretan opis teme.
\item Da li sažetak sadrži prave podatke o radu?\\
Sažetak sadrži prave podatke o radu.
\item Da li je rad lak-težak za čitanje?\\
Rad je lak za čitanje.
\item Da li je za razumevanje teksta potrebno predznanje i u kolikoj meri?\\
Nije potrebno predznanje.
\item Da li je u radu navedena odgovarajuća literatura?\\
Nedostaje bar jedan naučni članak iz odgovarajućeg časopisa, ali ne bih tu zamerala puno zbog same teme.
\odgovor{Prihvaćeno i ispravljeno - Dodat naučni članak u Python Belgrade sekciji.}
\item Da li su u radu reference korektno navedene?\\
Reference su korektno navedene.
\item Da li je struktura rada adekvatna?\\
Struktura rada je adekvatna.
\item Da li rad sadrži sve elemente propisane uslovom seminarskog rada (slike, tabele, broj strana...)?\\
Rad sadrži bar jednu sliku, tabelu i broj strana između 10 i 12.
\item Da li su slike i tabele funkcionalne i adekvatne?\\
Jesu.
\end{enumerate}

\section{Ocenite sebe}
% Napišite koliko ste upućeni u oblast koju recenzirate: 
% a) ekspert u datoj oblasti
% b) veoma upućeni u oblast
% c) srednje upućeni
% d) malo upućeni 
% e) skoro neupućeni
% f) potpuno neupućeni
% Obrazložite svoju odluku
Srednje upućena u oblast. Čitala sam o nekim od navedenih organizacija i temama kojima se bave, ali nisam nikad prisustvovala predavanju niti  bila na nekom od okupljanja.


\chapter{Dodatne izmene}
%Ovde navedite ukoliko ima izmena koje ste uradili a koje vam recenzenti nisu tražili. 
\odgovor{Posećenost na okupljanjima možete videti na grafiku 2 -> Posećenost na okupljanjima možete videti na slici 2}

\end{document}

