\documentclass[hyperref={bookmarks=false},aspectratio=169]{beamer}
\usepackage[utf8]{inputenc}
\usepackage[english,serbian]{babel}
\usepackage{color}
\usepackage{url}
\usepackage[unicode]{hyperref}
\hypersetup{colorlinks,citecolor=green,filecolor=green,linkcolor=blue,urlcolor=blue}

% ---------------  Define theme and color scheme  -----------------
\usetheme[sidebarleft]{Caltech}  % 3 options: minimal, sidebarleft, sidebarright

%\setbeamertemplate{footline}[frame number]

% ------------  Information on the title page  --------------------
\title[]
{\bfseries{Udruženja koja organizuju redovna IT okupljanja u Beogradu}}

\subtitle{Seminarski rad u okviru kursa\\Metodologija stručnog i naučnog rada}

\author[]
{Đorđe Vujinović, Nebojša Koturović, \\Igor Radojević, Bojan Stefanović}

\date[ICUP, 2014]
{Matematički fakultet, Univerzitet u Beogradu\\29. April 2020.}
%------------------------------------------------------------




\begin{document}

\frame{\titlepage}  % Creates title page


\section{Uvod}

%---------------------------------------------------------
%Changing visivility of the text
\begin{frame}
\frametitle{Uvod}

\begin{itemize}
	%\item Susret (eng. \textit{meetup}) okupljanje ljudi sa istim interesovanjem za neku određenu temu
    \item Susret se obično sastoji od nekog predavanja, radionice i dela za druženje i osveženje
    \item Predstavljeni ciljevi, mesto održavanja sastanaka, razlozi održavanja i istorijati organizacija
    \item Cilj prezentacije je upoznati čitaoca sa udruženjima
    \item PHP Srbija, WordPress Serbia, Python Belgrade, RISK, Belgrade Crypto Community, Agile Serbia i JS Belgrade
    %\item Dodatne organizacije
\end{itemize}

\end{frame}

\section{PHP Srbija}

%---------------------------------------------------------
%Highlighting text
\begin{frame}
\frametitle{PHP Srbija}

\begin{columns}[T]

\column{0.75\textwidth}
\begin{itemize}
    \item PHP je popularan skriptni jezik za izradu dinamičkih veb stranica 
    \item Cilj udruženja je okupljanje stručnjaka i PHP programera u promovisanju novih tehnologija u samom jeziku
    \item Prvo okupljanje u julu 2015. godine
    \item Predavači su najčešće PHP programeri i sistem administratori
    \item Pored okupljanja, udruženje organizuje i konferencije
\end{itemize}

\column{0.25\textwidth}
\begin{figure}
    \raggedleft
    \includegraphics[scale=0.15]{./images/php_gray.jpg}
\end{figure}

\end{columns}

\end{frame}
%---------------------------------------------------------

\section{WordPress Serbia}

%---------------------------------------------------------
%Changing visivility of the text
\begin{frame}
\frametitle{WordPress Serbia}

WordPress predstavlja jedan od najpopularnijih sistema za upravljanje sadržajem (eng. \emph{Content Management System - CMS}).

\begin{columns}[T]
\column{0.75\textwidth}
\begin{itemize}
    %\item 
    \item Udruženje predstavlja glavnu i centralizovanu grupu WordPress zajednice u Srbiji
    \item Prvo okupljanje u aprilu 2013. godine
    \item Teme na okupljanjima prilagođene početnicima, programerima, kao i ostatku IT zajednice
    \item Udruženje takođe organizuje WordCamp konferencije
\end{itemize}


\column{0.25\textwidth}

\begin{figure}
    \raggedleft
    \includegraphics[scale=0.4]{./images/wp.jpg}
\end{figure}

\end{columns}
\end{frame}

\section{Python Belgrade}
%---------------------------------------------------------
%Changing visivility of the text
\begin{frame}
\frametitle{Python Belgrade}
Python Belgrade je organizacija koja organizuje okupljanja za sve Python entuzijaste. Pre svega programere, ali i sve one zainteresovane za ovaj jezik.

\begin{columns}[T]
\column{0.75\textwidth}

\begin{itemize}
    \item Osnovana 2015. godine, na čelu sa Bojanom Jovanovićem
    \item Okupljanja se održavaju svakih par meseci
    \item Cilj okupljana je razmena znanja vezanih za jezik Python
    \item Najčešća mesta održavanja ICT HUB i StartIt Centar
    \item Informacije o okupljanjima su dostupne na veb-sajtu organizacije i socijalnim mrežama
\end{itemize}

\column{0.25\textwidth}
\begin{figure}
    \raggedleft
    \includegraphics[scale=0.075]{./images/pybgd.png}
\end{figure}

\end{columns}

\end{frame}

\section{RISK}
%---------------------------------------------------------
%Changing visivility of the text

\begin{frame}
\frametitle{RISK}

%\column{0.25\textwidth}
\vspace{-8mm}
\begin{figure}
    \raggedright
    \includegraphics[scale=0.060]{./images/riskmatf.png}
\end{figure}

RISK je studentska organizaciju Matematičkog fakulteta koja za cilj ima 
okupljanje i edukaciju studenata zainteresovanih za oblast računarstva i informatike.

% \begin{columns}[T]
% \column{0.75\textwidth}

\begin{itemize}
    \item Osnovano u decembru 2016. godine na čelu sa Nemanjom Mićovićem
    \item Okupljanja se organizuju u zgradi Matematičkog fakulteta
    \item Teme predavanjima su vezane za programiranje (radni okviri, programerski alati ...) 
    \item Informacije o okupljanjima su dostupne na veb-sajtu i Facebook grupi organizacije
\end{itemize}

% \end{columns}
\end{frame}


\section{Belgrade Crypto Community}

\begin{frame}
\frametitle{Belgrade Crypto Community}

\begin{columns}[T]

\column{0.75\textwidth}
\begin{itemize}
    \item Kriptovaluta, blokčejn, majner, digitalni novčanik i bitkoin
    \item Prvo okupljanje održano februara 2017. godine
    \item Susreti su namenjeni svima od korisnika i novajlija do iskusnih programera blokčejna
    \item Susreti svakog drugog četvrtka u mesecu
    \item Pored toga, organizuju i radionice
\end{itemize}


\column{0.25\textwidth}

\begin{figure}
    \raggedleft
    \includegraphics[scale=0.25]{./images/bcc_logo.png}
\end{figure}

\end{columns}
\end{frame}


\section{Agile Serbia}

\begin{frame}
\frametitle{Agile Serbia}

Agilni razvoj softvera je skup okvira i metoda razvoja softvera na principima iterativnog i inkrementalnog razvoja, gde zahteve korisnika u rešenja implementiraju samoorganizovani razvojni timovi.


\begin{columns}[T]

\column{0.75\textwidth}
\begin{itemize}
    \item Agilni Manifesto
    \item Cilj agilni razvoj softvera i njegovo unapređenje
    \item Prvo okupljanje održano 2012. godine
    \item Predavači su ljudi sa dugogodišnjim iskustvom iz zemlje i inostranstva
    \item Udruženje takođe organizuje Agile Serbia konferencije 
\end{itemize}


\column{0.25\textwidth}

\begin{figure}
    \raggedleft
    \includegraphics[scale=0.4]{./images/agile_srb.png}
\end{figure}

\end{columns}
\end{frame}

\section{JS Belgrade}

%---------------------------------------------------------
%Changing visivility of the text
\begin{frame}
\frametitle{JS Belgrade}

\begin{columns}[T]

\column{0.75\textwidth}
\begin{itemize}
    \item JavaScript je skriptni programski jezik napravljen za veb programiranje na klijentskoj strani
    \item Udruženje osnovano 2014. godine od strane Slobodana Stojanovića, Bogdana Gavrilovića i Alexandera Simovića
    \item Cilj povezivanje ljudi zainteresovanih za JavaScript
    \item Prvo okupljanje u decembru 2014. godine
    \item Do sada organizovano 36 okupljanja
\end{itemize}

\column{0.25\textwidth}

\begin{figure}
    \raggedleft
    \includegraphics[scale=0.2]{./images/JS_logo.png}
\end{figure}
\end{columns}
\end{frame}

\begin{frame}
\frametitle{JS Belgrade}

\begin{columns}[T]

\column{0.75\textwidth}
\begin{itemize}
    \item Najčešća mesta održavanja ICT HUB i Cloud Horizon, a pored toga Matematički fakultet, Crowne Plaza Belgrade, StartIt Centar itd.
    \item Slušalac i predavač na okupljanju
    \item Druženja u kafeteriji Koffein %počev od 19. januara 2016. godine
    \item Gostovanja van Beograda %(Niš 19. marta 2016. i Inđija 26. marta 2016.)
\end{itemize}

\column{0.25\textwidth}

\begin{figure}
    \raggedleft
    \includegraphics[scale=0.2]{./images/JS_logo.png}
\end{figure}
\end{columns}
\end{frame}

\section{Zaključak}

%---------------------------------------------------------
%Changing visivility of the text
\begin{frame}
\frametitle{Zaključak}

\begin{itemize}
    \item Beograd nudi dosta zanimljivih sadžaja ljudima iz IT sveta
    \item Primetno je povećanje događaja ove vrste
    \item Potrebno je blagovremeno se informisati o predstojećim okupljanjima
\end{itemize}

\end{frame}

\section{Zahvalnost i pitanja}

%---------------------------------------------------------
%Changing visivility of the text
\begin{frame}
\frametitle{Zahvalnost i Pitanja}

\begin{center}
\Huge{Hvala na pažnji!}
\\ \\
\LARGE{Pitanja?}
\end{center}

\end{frame}

\section{Literatura}
%---------------------------------------------------------
%Changing visivility of the text
\begin{frame}
\frametitle{Literatura}

%\url{https://phpsrbija.rs/} \\
%\small{\url{https://phpsrbija.rs/}} \\
%\footnotesize{\url{https://phpsrbija.rs/}}  \\
%\scriptsize{\url{https://phpsrbija.rs/}}

\footnotesize{\url{https://phpsrbija.rs/}} \\
\footnotesize{\url{https://sr.wordpress.org/}} \\
\footnotesize{\url{https://www.meetup.com/WordPress-Serbia/}} \\
\footnotesize{\url{http://www.admissions.caltech.edu/pranks}} \\
\footnotesize{\url{https://www.tiobe.com}} \\
\footnotesize{\url{https://www.meetup.com/PythonBelgrade}} \\
\footnotesize{\url{http://risk.matf.bg.ac.rs}} \\
\footnotesize{\url{https://belgradecrypto.com/}} \\
\footnotesize{\url{https://www.meetup.com/Belgrade-Crypto-Community/}} \\
\footnotesize{\url{http://www.agile-serbia.rs/}} \\
\footnotesize{\url{http://jsbelgrade.org/}} \\
\footnotesize{\url{https://www.meetup.com/JS-Belgrade-Meetup/}} \\


\end{frame}

%---------------------------------------------------------
%Two columns
% \begin{frame}
% \frametitle{Hollywood sign}
% 
% \begin{columns}
% 
% \column{0.45\textwidth}
% 
% % \begin{figure}
% %     \centering
% %     \includegraphics[width=\columnwidth]{./figures/hollywood_caltech.jpg}
% %     \caption{``Hollywood is still mad about that,'' says Autumn Looijen, author of \emph{Legends of Caltech III: Techer In the Dark.} \tinimagesy{(Photo downloaded from: http://brennen.caltech.edu/autobiography/automaster2.htm)}}
% %     \label{fig:hollywood_prank}
% % \end{figure}


% \column{0.55\textwidth}
% In May 1987, undergraduates from Page and Ricketts houses combined forces (and several hundred dollars) to purchase enough black and white plastic, transformed the Hollywood sign to read ``Caltech''.

%\end{columns}
%\end{frame}
%---------------------------------------------------------
\end{document}
